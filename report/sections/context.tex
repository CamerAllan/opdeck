\chapter{Context Survey}

This project aims to develop a framework for capturing the opinions of its users. This closely aligns with the idea of gamifying surveys - a concept that is introduced in Section~\ref{surgam}. Section~\ref{soft} describes existing products that attempt to fulfil a similar role to OpDeck, analysing their various strengths and weaknesses.

\section{Survey Gamification}\label{surgam}
An emerging trend in modern business practice, \textit{gamification} can be defined as `using game mechanics and game design elements to measure, influence and reward user behaviors.'~\cite{7804551}. 
There has been some research into gamification as a tool to improve the quality of responses from and engagement with surveys. From this I aim to gain an awareness of the necessary considerations for designing and implementing gamification.

As of yet, most experiments into survey gamification have only gone so far as to change the wording of survey questions, show imagery alongside questions or change the answer input mechanics.
The objectives of \od{} represent a considerable departure from the standard survey format, with the goal being that the game can be enjoyed as a standalone experience. 
As this is somewhat uncharted territory, there are many psychological considerations as to the validity of the results that can be gathered through this new format. 
There exists a tradeoff between complex levels of gamificaiton that increase enjoyment, and the accuracy of respondent data. 
It is also difficult to be certain whether players respond the same way in a game as they might in real life.
While deep investigation of these issues is not my goal, I deem it important to understand them, in an attempt to minimise any bias that the \od{} framework could impose onto players.

A 2011 paper found that while gamification generally increases participants' enjoyment of surveys, there are three main aspects of gamification that can affect the `character of the data'~\cite{GameExperiments}:
\begin{enumerate}[label=\textbf{e.\arabic*}]
    \item\label{cd:q} Effects caused by changes to the question and how it is interpreted
    \item\label{cd:m} Effects caused by changing the attitude and mindset of respondents
    \item\label{cd:l} Effects caused by changes to the design and layout of question
\end{enumerate}
While these considerations are targeted toward those gamified surveys that have a fairly standard format, I believe that they are still applicable to \od. 

A key aspect of \od{}'s gameplay loop is the context that persists between questions, affecting players' choices. Harms \etal{} wrote in a 2015 paper~\cite{Olympic} that `Designers may also implement feedback loops, i.e., dynamics wherein user actions affect the overall state of gameplay. Feedback loops may visualize concepts such as a user’s progress, status, wealth, health, points, etc.'. 
This gives me confidence that adding a persistent context to the game is a valid design decision.
As for how this will affect the character of the data, I believe this relates to~\ref{cd:q} (question/interpretation). 
This is because changing the context effectively changes the question, as people may respond differently under different circumstances. 

\ref{cd:l} (design/layout) applies not just to \od, but to any presentation of a survey - where there exists an interface to interact with, there is potential for it to influence the way the user responds to it. I attempt to minimise this risk - ideally the only bias imposed on a player's decision is from the game definition itself.

Brownell \etal{}~\cite{SurveyGamificationResearch} summarise the state of research into survey gamification as follows:
`In most cases, their results show that the addition of these game elements increases the length and quantity of responses, and respondents typically prefer the gamified version to the standard survey version. However, their research does not compare the effectiveness of game elements in gamified surveys. They have also found that that some gamified survey designs can lead to compromised respondent data (Puleston \& Sleep, 2011).'~\cite{SurveyGamificationResearch}.
This indicates that there is promise in the notion of gamifying surveys - however it can result in a reduction in accuracy of the data collected. Depending on the extent to which the survey is abstracted from, precise details, both of the question and response, can be lost. Given this, gamification is likely best suited to situations in which the surveyors need broad answers, rather than precise ones.

\section{Similar Software}\label{soft}

\subsection{Datagame}
Datagame~\cite{Datagame} is a company that offers services allowing researchers to create and publish gamified surveys. This is done through recognisable interfaces that are heavily game-influenced, such as word searches and decks of cards.

Datagame supplies the tools required to populate one of several template games with custom questions, and export this to an online format that can be played through multiple channels including Facebook~\cite{Facebook}. 

Datagame offer four customisable game types. That most similar to \od{} is referred to as a `Swiper' game. In this, the player draws cards from a virtual deck, each of which presents a yes/no question. The user swipes the card left or right to answer yes or no respectively. Figure~\ref{fig:datagame} shows this interface - it is fairly simple, with no standout features other than those added during configuration. Customisation options include allowing the user to replace card backgrounds with images, as well as change the colour of the text. Figure~\ref{fig:dg_editor} shows an example of the game editing view in which the following aspects of the game can be edited:

\begin{itemize}
    \item Project title
    \item Title question
    \item List of cards
    \item Card name
    \item Card text/images
    \item Toggle card shuffling
    \item Game UI labels
    \item Background image
\end{itemize}

\begin{figure}[!h]
	\centering
	\includegraphics[width=0.9\textwidth]{./images/context/datagame.png}
	\caption{Example question in one of the Datagame~\cite{Datagame} game types, which involves answering yes or no questions by swiping left or right.}
	\label{fig:datagame}
\end{figure}

\begin{figure}[!h]
	\centering
	\includegraphics[width=0.9\textwidth]{./images/context/dg_editor.png}
	\caption{Process of editing the Datagame~\cite{Datagame} game instance shown in figure~\ref{fig:datagame}.}
	\label{fig:dg_editor}
\end{figure}

This format excels in allowing participants to answer a high volume of questions in a short period of time. 
The participant is asked questions directly, therefore there should be no ambiguity in their understanding beyond the wording of individual questions. 
The limitation of a binary choice reduces the precision with which players can answer a given question - this may be acceptable for only a subset of surveys not requiring precise answers.

While the interface has the appearance of a game, the gameplay elements are shallow. 
There is no conditional branching in this game, as cards can only be removed by answering them and there is no way to add cards to the deck during a game. 
Further, the participant's responses have no observable consequences on the future of the game and there is no ultimate goal. This leaves the player with little incentive to continue playing beyond the appearance of the interface, and the content of the questions themselves.

Given the above, while Datagame surveys are presented as games, at their core they are merely a wrapper around a basic binary choice survey format.

\subsection{Qualifio}
Qualifio~\cite{Qualifio} offer a similar service, but provide a larger variety of game formats than Datagame. 
I was not able to access Qualifio's game creation tools, as this functionality is behind a paywall, however the site does host playable examples. 
Qualifio's toolset features a `Swiper' style game similar to  that provided by Datagame, which can be seen in figure~\ref{fig:qual}. This comes with the same tradeoffs between engagement, accuracy of results and precision of input as the Datagame equivalent.

\begin{figure}[!h]
	\centering
	\includegraphics[width=0.9\textwidth]{./images/context/qual.png}
	\caption{Qualifio~\cite{Qualifio} swiper demo. This example game has players swipe or click depending on which team they think a given football player belongs to.}
	\label{fig:qual}
\end{figure}

Figure~\ref{fig:unit} shows another game template offered by Qualifio, named `Ranking'. 
Ranking provides a different approach to gathering user opinions, where multiple options are ranked in order of preference.
This is done through a click and drag interface - more engaging and intuitive than the pen and paper equivalent of labelling each option with a rank number. 

\begin{figure}[!h]
	\centering
	\includegraphics[width=0.9\textwidth]{./images/context/united.png}
	\caption{Qualifio~\cite{Qualifio} Ranking demo. This example has players rank football players from top to bottom - best to worst.}
	\label{fig:unit}
\end{figure}

Ranking gives users more precision in their answers than is available with Swiper, however the results within a question are all relative to each other. If one user likes all of the options, and another dislikes them all, it's plausible that they will give the same answer - the absolute values are lost in place of relative data.

Figures~\ref{fig:check} and~\ref{fig:pred} show examples of other game templates that Qualifio offer.

\begin{figure}[!h]
	\centering
	\includegraphics[width=0.9\textwidth]{./images/context/check.png}
	\caption{Qualifio~\cite{Qualifio} Checklist demo. Players check answers that they agree with.}
	\label{fig:check}
\end{figure}

The prediction game in figure~\ref{fig:pred} provides an example of a game format that targets a specific category of questions - what players predict the result of an upcoming competition will be.

\begin{figure}[!h]
	\centering
	\includegraphics[width=0.9\textwidth]{./images/context/pred.png}
	\caption{Qualifio~\cite{Qualifio} Prediction demo. Players predict the scores of two parties.}
	\label{fig:pred}
\end{figure}

\subsection{Reigns}
Reigns is a multi-platform game in which the player takes the role of a medieval ruler, making binary decisions to solve problems their subjects approach them with. 
These decisions affect which scenarios may later appear, as well as changing how the ruler is perceived by various factions such as their population, army, and church. 
The player's success is measured by how many decisions they can make without falling out of favour with any of the factions.

Reigns is marketed purely as a game, with no data-collecting functionality. In addition, the framework on which it is built is proprietary, so users are not able to design game scenarios around topics of their own interest. For these reasons, Reigns cannot reasonably be used to collect and analyse data from players' choices. 

As of 2019-03-01, Reigns is a well reviewed game, with a rating of 4.7/5 on the Google Play store~\cite{GooglePlay}.
Building intuitive creation, play, and analysis platforms for games in the style of Reigns has the potential to make rich gamification of surveys accessible to a wider research audience.
