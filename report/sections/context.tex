\chapter{Context Survey}

Here I will review the existing software and literature on the different aspects of this project. 
I will briefly look at the psychological and gamification aspects 
\todo{finish}

\section{Opinion of First Contact?}
As this project has been undertaken in collaboration with the Exoplanet Research Society, it is worth briefly reviewing the literature on the target question - how would humans react to the discovery of alien life? There has been some research\cite{ReactionsToMessage, HowReactDiscovery, Fear} into this area, but it is certainly not extensive and very little of it is systematic. The topic is vast with many variations:
\begin{itemize}
    \item How has extra-terrestrial life been discovered?
    \item What is the nature of the life, is it intelligent?
    \item Is alien life on our doorstep, or far enough away that it could never reach or harm us?
\end{itemize}
Anne Smith and Christine Helling believe it could be beneficial to create a game capable of collecting this information, in order to reduce the burden on participants answering such an extensive library of questions and improve engagement.
\todo{finish}

\section{Similar Software}
\todo{fill}

\section{Survey Gamification}
An emerging trend in modern business practice, gamification consists of `using game mechanics and game design elements to measure, influence and reward user behaviors.'~\cite{7804551}. 
There has been some research into gamification as a tool to improve the quality of responses from and engagement with surveys. 
The results of this research has been summarised by Briana Brownell, Jared Cechanowicz, and Carl Gutwin as follows: 
`In most cases, their results show that the addition of these game elements increases the length and quantity of responses, and respondents typically prefer the gamified version to the standard survey version. However, their research does not compare the effectiveness of game elements in gamified surveys. They have also found that that some gamified survey designs can lead to compromised respondent data (Puleston \& Sleep, 2011).'~\cite{SurveyGamificationResearch}.
This indicates that there is promise in the notion of gamifying surveys.
As noted by Brownell et al, however, gamification can result
\todo{finish}

Typically, the experiments discussed in the papers referenced only go as far as to change the wording of survey questions, show imagery or 
\todo{finish}

A 2011 paper found that while participants' enjoyment of the survey sees a great increase, there are three main effects that gamification has on the `character of the data'~\cite{GameExperiments}:
\begin{enumerate}
    \item Effects caused by changes to the question and how it is interpreted
    \item Effects caused by changing the attitude and mindset of respondents
    \item Effects caused by changes to the design and layout of question
\end{enumerate}
These are things I kept in mind while making design decisions, as if the software is to be used in an experiment, it's imperative that there is no bias inherent in the framework.

\section{Game Model}
\subsection{Reigns}
When proposing the format of the artifact, I based my framework on an existing game, Reigns~\cite{Reigns}. 
In Reigns, the player takes the role of a medieval ruler, and makes binary decisions to solve problems their subjects approach them with. 
These decisions affect further scenarios that may appear, as well as changing how the ruler is perceived by various factions, such as their population, army and church. 
The player's success is measured by how many decisions they can make without falling out of favour with any of the factions.

As of 2019-03-01, this is a well reviewed game, with a rating of 4.7/5 on the Google Play store~\cite{GooglePlay}. 
Given this, in addition to the simplicity of recording and analysing the choices, I believe the framework of Reigns could provide a good starting point for gathering player opinions. 
