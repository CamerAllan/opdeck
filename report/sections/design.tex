\chapter{Design}

\section{Gameplay Overview}
\od{}'s game UI provides a fairly straightforward experience from the player's point of view, however there is a significant amount of complexity that the user does not see.
Conceptually it is a card game, so it will be described as such.
The game consists of three decks which I dub `in play', `out of play' and `reserve'. In addition to this, there are $n$ `pillars' - these are attributes of importance within the context of the game story. Each of these has a minimum, maximum and current value.
The `in play' deck has a defined starting set of cards, with the `reserve deck' containing all others - `out of play' starts empty.
Both decks are shuffled at the beginning of the game, and each pillar has a predefined starting value.

The player draws and reads a card from the play deck, each one showing the following information:

\begin{itemize}
	\item Title
	\item Description
	\item Choice \#1 (`accept')
	\item Choice \#2 (`reject')
	\item Requirements to draw
\end{itemize}
Each choice on a card consists of text detailing the response, and the effects of the response. Effects are made up of two parts - pillar changes, and cards added/removed. Pillar changes specify amounts to add or remove from one or more pillars. Cards added/removed defines which cards should be moved between `out of play' and `reserve'.
`Requirements to draw' consists of conditions that the current pillar levels must satisfy in order for this card to be drawn, such as the current value being within a certain range.

After each choice made (a turn), any cards in `out of play' whose pillar requirements are met are moved to `in play', while any `in play' whose requirements are not met are moved to `out of play'.

The game ends when any of the pillars fall to their minimum value.

\begin{figure}[!h]
	\centering
	\includegraphics[width=0.9\textwidth]{./images/design/diag.png}
	\caption{Diagram of the main game loop. Actions followed by a question mark are optional, depending on the game definition}
	\label{fig:diag}
\end{figure}

\section{UI}

\subsection{Game}
As it is described above, the game takes a lot of effort on behalf of the player, having to sort and shuffle cards. Fortunately, this effort can be removed completely through work done by the computer. This leaves a simple game from the user perspective; users are presented with a card, make their choice, and get the next card.

This simple perspective means that the design for the game UI is quite straightforward. Figure~\ref{fig:game_comp} shows the first design for this UI alongside the final product. Initially, it was uncertain whether or not the pillars would be visible to the player. It is generally accepted that visual feedback is an important aspect of interface design~\cite{fb}. An additional, important consideration in this decision was the  edge case where a card may present itself on two consecutive turns. Taking these points into account, it was decided that visualising pillar updates was important, so this was included at the top of the interface.

\begin{figure}[!h]
	\centering
	\includegraphics[width=0.36\textwidth]{./images/design/game_drawing.jpg}
	\includegraphics[width=0.54\textwidth]{./images/design/game.png}
	\caption{Comparison between initial and final designs for the game UI}
	\label{fig:game_comp}
\end{figure}

Pillars are displayed as vertical bars at the top of the screen. 
These fill from bottom to top, representing where the current value lies between the predefined min and max.
Each pillar has its own fill colour (customisable in the game maker), which was inspired by Datagame~\cite{Datagame} who offer the same feature for their Swiper game. This serves to make the game more visually engaging without imposing any particular colour scheme, as doing so could conflict with the desires of game creators.

\begin{figure}[!h]
	\centering
	\includegraphics[width=0.9\textwidth]{./images/design/pillars.png}
	\includegraphics[width=0.9\textwidth]{./images/design/hover.png}
	\caption{Visualisation of a negative outcome; the red outline around \c{Popularity} }
	\label{fig:colour_comp}
\end{figure}

The image shown with a given card is dependent upon the pillars that the card influences, as seen in figure~\ref{fig:advisor_comp}. In this example game, each pillar was assigned an `advisor' image, so that the advisor representing the pillar most affected by a card is shown.

\begin{figure}[!h]
	\centering
	\includegraphics[width=0.45\textwidth]{./images/design/game.png}
	\includegraphics[width=0.45\textwidth]{./images/design/ecman_blue.png}
	\caption{These cards primarily affect different pillars, so they have different associated images}
	\label{fig:advisor_comp}
\end{figure}

As seen in figure~\ref{fig:colour_comp}, not only can the image change, but the primary colour of the image may also change. This feature is included as cards may affect multiple pillars, which would otherwise not be represented. With this colour shifting implemented, images can represent secondary pillar effects.

\begin{figure}[!h]
	\centering
	\includegraphics[width=0.45\textwidth]{./images/design/ecman_green.png}
	\includegraphics[width=0.45\textwidth]{./images/design/ecman_blue.png}
	\caption{Both of these cards primarily affect the \c{Money} pillar (they are being presented by the \c{Money} advisor) however the second also affects \c{Popularity}, so has a blue primary colour}
	\label{fig:colour_comp}
\end{figure}

\subsection{Admin tools}

\subsubsection{Game Maker}
The game maker interface was the most challenging to design, as I wanted the user to be able to maintain a high-level overview of the game while adding and editing pillars and cards.
After thinking this through, I initially settled on the design depicted in figure~\ref{fig:game_maker}. The main theme is that editing is done in the left panel, while the right side continues to show an interactive view of the game, including a visualisation of the relationships between cards.
The final design is roughly the same, however the card view is not as complex as the connections between cards are not visualised. 
There are two ways in which cards can be connected (Add to deck/Remove from deck), and further still, a card may be connected to an arbitrary number of others in these ways. For this reason the final design displays cards in a grid instead of a graph.

\begin{figure}[!h]
	\centering
	\includegraphics[width=0.36\textwidth]{./images/design/game_maker_drawing.png}
	\includegraphics[width=0.54\textwidth]{./images/design/game_maker.png}
	\caption{Comparison between initial and final designs for the game maker. Shown is the example game I created for demonstration purposes.}
	\label{fig:game_maker}
\end{figure}

In the right hand section of the interface, the user can see the game pillars and cards that they have created. This side is scrollable, while the left section remains in view. If the details of a card are too large for the view, the text fades out. The card view can be expanded by hovering over it, as seen in figure~\ref{fig:expand} - when this is done, other cards shrink and move to account for the extra space needed. New cards and pillars can be added with the large \c{+} buttons by the headers, while existing ones can be edited by clicking on them. Both of these options open the edit view in the left hand section.

\begin{figure}[!h]
	\centering
	\includegraphics[width=0.45\textwidth]{./images/design/cards_not_expanded.png}
	\includegraphics[width=0.45\textwidth]{./images/design/cards_expanded.png}
	\caption{Demonstration of the hovering expanded view (note that the full text is cut off on the left, readable on the right)}
	\label{fig:expand}
\end{figure}

The left hand section provides contextual menus. When no pillar or card is selected, it provides high-level functionality, such as saving, switching, or creating new game definitions. Also present here is starting deck selection, game balance information and any warnings.

`Pillar effect balance' provides the admin user with an idea of how balanced their game is in its current state. This is done by summing all of the effects that are applied to each pillar for any given response to all cards. This acts as an indicator of how well a user is likely to do if they randomly pick their responses; the higher the balance values, the easier the game. In addition, these figures serve to notify the user of any unintentional bias within their game. Understanding and manipulating the relationship between cards and pillars is key in gathering unbiased data.

Warnings appear when a card has no chance of being added to the game. This means that a card has been created, but it is neither in the starting deck, nor is it added by any other card. This is only a shallow check - if card A is added by another card B, which itself is never added, the warning will only appear for card B. I consider this acceptable, as once card B is added or deleted, the situation will be resolved.

\begin{figure}[!h]
	\centering
	\includegraphics[width=0.6\textwidth]{./images/design/info.png}
	\caption{Info on left panel. Note the two warnings, which appeared on removing \c{recruitmentPaidOff} and \c{loanRepay} from all `cards added' lists}
	\label{fig:info}
\end{figure}

When editing a card, a live preview of its contents can be seen at the bottom of the panel, alongside buttons to cancel changes, delete, or submit the card. When editing the consequences of a response, adding and removing cards is done through a dropdown list that allows multiple cards to be selected.

\begin{figure}[!h]
	\centering
	\includegraphics[width=1.0\textwidth]{./images/design/card_edit.png}
	\caption{Card editing view, reached by selecting a card by clicking it.}
	\label{fig:card_edit}
\end{figure}

\begin{figure}[!h]
	\centering
	\includegraphics[width=1.0\textwidth]{./images/design/add_card.png}
	\caption{Dropdown showing all other cards that can be added or removed}
	\label{fig:add_card}
\end{figure}

Figure~\ref{fig:pillar_edit} shows the pillar editing view, which is very similar to the card editing view. The colour of the pillar can be entered as a hex value, which is immediately reflected in the preview below.

\begin{figure}[!h]
	\centering
	\includegraphics[width=1.0\textwidth]{./images/design/pillar_edit.png}
	\caption{Pillar editing view}
	\label{fig:pillar_edit}
\end{figure}

Clicking \c{Save Current} will save the updated game definition to the configured game definition database, provided it is running. Once saved, a game can immediately be played from the game UI by entering the corresponding game ID.

\subsubsection{Visualisation}

The visualisation application is of similar form to the game maker, with data selection and filtering happening in a left panel while the visualisations are updated on the right.
It is possible to adjust the data used in the visualisations by filtering the pillars by value. This limits the data shown only to cards that appeared while pillar values match the specified criteria. The visualisations I chose to show are as follows:
\begin{itemize}
	\item Accept/Reject balance

	      This is a horizontal bar chart showing the proportion of players that choose one option over the other for a given set of cards. Each card has a value between -1 and 1, where -1 indicates that players choose the reject option every time, while 1 indicates accept is chosen every time.

	\item Total times drawn

	      This is a vertical bar chart showing the total number of times each card has been drawn and responded to.

	\item Pillar average

	      This shows the average pillar levels over all turns.
\end{itemize}

The intention of these visualisations is to allow an admin user to `play' with the data. The live updating charts allow for the user to rapidly identify relationships in the data, for example between pillar levels and accept/reject balance.

\begin{figure}[!h]
	\centering
	\includegraphics[width=0.45\textwidth]{./images/design/visualisation.png}
	\includegraphics[width=0.45\textwidth]{./images/design/visualisation_filter.png}
	\caption{Visualisation screen effects of filtering data by cards and pillar values. This example filters responses to cards during turns where the player's \c{Army} was between 0 and 5, and their \c{Popularity} was between 5 and 10.}
	\label{fig:visualisation}
\end{figure}

The other function offered by this page is the ability to export data to CSV. This effectively repackages the contents of the user database as a csv file, where each row represents the turn of a user. Individual games can be uniquely identified by sets of rows with the same user id and game id.