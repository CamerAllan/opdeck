\chapter{Introduction}

\section{Motivation}
This project is undertaken in collaboration with the St Andrews Centre for Exoplanet Science, with an aim to answer the question; how would the general human population react to the discovery of alien life?
There has been research~\cite{ReactionsToMessage, HowReactDiscovery, Fear} into this area, but it is certainly not extensive and very little of it is systematic. The topic is vast, and is made more complex by the many factors that may affect the answer:
\begin{itemize}
    \item How has extra-terrestrial life been discovered?
    \item What is the nature of the life, is it intelligent?
    \item Is alien life on our doorstep, or far enough away that it could never reach or harm us?
\end{itemize}
Anne Smith (Biology) and Christine Helling (Astronomy) of the St Andrews Centre for Exoplanet Science believe it could be beneficial to create a game capable of collecting answers to questions like these. Gamification has been used in many areas, and has been shown to consistently improve engagement~\cite{engage}. Hopefully, this format of gathering opinions would allow participants to become immersed in the context of the quesiton, leading to more honest responses. 

If it was possible to accurately capture players' thoughts and feelings in this manner, this could impact the way opinion data is collected across multiple faculties.

\section{Achievements}
The \od{} framework can be conceptualised into three distinct sections:
\begin{itemize}
    \item Game Interface
    \item Game Maker
    \item Visualisation
\end{itemize}
The game interface provides a portal through which participants can play games built and stored on the platform. The game maker provides an interface through which an administrator may create and edit games. Finally, the visualisation tools allow administrators to filter and visualise how players interact with the game.

These parts work together to form a pipeline that enbles an administrator to design a game that explores their topic of interest, which can then be played by any number of users. Following this, the admin may explore player's choices using the visualisation and filtering tools provided, or export the data to perform more extensive analysis.  

\section{Document Outline}
After outlining the objectives of this \od{}, this document reviews the software and literature surrounding gamification - the primary area of research with which the project is concerned.
Covered then are the processes followed, tools used and considerations taken into account when designing, implementing and testing the framework. 
This is followed by an analysis of user testing and evaluation.
Finally, the software is evaluated with respect to the original objectives.

\section{Objectives}
As part of my Description, Objectives, Ethics and Resources (DOER) document I describe the objectives of this project. Over the course of the project, they remained the same apart from one change - downgrading~\ref{SO:2} to being a secondary objective, from being a primary objective. This was done as the focus of the project shifted towards creating an unbiased framework, while this objective was content-focused and therefore became lower priority.

\subsection{Primary Objectives}
\begin{enumerate}[label=\textbf{PO.\arabic*}]
    \item Devise and implement a game that presents the player with scenarios and allows them to choose from potential responses
    \item Devise and implement a flexible infrastructure to model and constrain scenarios and their impacts
    \item Devise and implement an infrastructure for capturing and recording player responses
    \item Implement basic visualisation of responses
\end{enumerate}
\subsection{Secondary Objectives}
\begin{enumerate}[label=\textbf{SO.\arabic*}]
    \item Devise and implement an admin centre to allow easy creation of new game content
    \item \label{SO:2} In collaboration with Anne Smith and Christine Helling, devise a sample set of appropriate scenarios with impacts and populate the game
    \item Carry out an experiment to assess the effectiveness of the game as a tool to assess people's real world views
    \item Create more advanced visualisation and analysis tools
\end{enumerate}
\subsection{Tertiary Objectives}
\begin{enumerate}[label=\textbf{TO.\arabic*}]
    \item Perform a wider user experiment
\end{enumerate}

\section{Definitions}
Throughout this text I will use the following words defined as so:
\begin{itemize}
    \item Player/Participant - A user playing the game through the UI
    \item Administrator - An user that has access to the game maker and visualisation applications. This would be an individual interested in researching the opinions a group holds on a subject matter.
    \item Game definition - A game within the context of the framework I have created. A unique game definition consists of a set of cards, a set of pillars, and a starting deck
\end{itemize}